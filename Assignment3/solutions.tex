\documentclass{article}
\usepackage[margin=1in]{geometry}
\usepackage{microtype}
\usepackage{amsmath}
\usepackage{graphicx}

\title{AMAT 491 Assignment 3}
\date{Nov 27 2017}
\author{Ian McKenzie}

\begin{document}
\pagenumbering{gobble}
\maketitle
\newpage
\pagenumbering{arabic}

% Question 1
\section{}
\subsection*{a.}
Let's first find the formula for the Composite Simpson's Rule.
\[Q_S^{c,n}(f)=\]
\subsection*{b.}
Now let's find the error estimate of this rule.
\subsection*{c.}
We now have our error estimate for the Composite Simpson's Rule as
\[\int_a^bf(t)df =  Q_s^{c,n}(f) - \frac{(b-a)^5}{2880n^4}f^{(4)}(\theta)\]
We want the absolute value of the error to be less than \(10^{-5}\)
over the interval [0,2]. This means we want to find n such that
\[\left|\frac{1}{90n^4}f^{(4)}(\theta)\right| \leq 10^{-5}\]
We are told that the maximum value of \(|f^{(4)}(\theta)|\) is 9, thus
we know
\[\left|\frac{1}{90n^4}f^{(4)}(\theta)\right|
\leq \left|\frac{1}{10n^4}\right| \leq 10^{-5}\]
We simply solve for n and find \(n\geq10\), thus we need to
perform at least 10 iterations to ensure accuracy. 



% Question 2
\section{}
See \texttt{composite\_simpson.m} and \texttt{question2.m}.

% Question 3
\section{}
To derive such a formula, we use Taylor theorem to write
\begin{itemize}
\item \(f(x_0+h) = f(x_0) + hf'(x_0) + \frac{h^2}{2}f''(x_0) +
  \frac{h^3}{6}f'''(x_0) + \frac{h^4}{24}f''''(x_0) +
  \frac{h^5}{120}f^{(5)}(\theta)\)
\item \(f(x_0-h) = f(x_0) - hf'(x_0) + \frac{h^2}{2}f''(x_0) -
  \frac{h^3}{6}f'''(x_0) + \frac{h^4}{24}f''''(x_0) -
  \frac{h^5}{120}f^{(5)}(\theta)\)
\item \(f(x_0+2h) = f(x_0) + 2hf'(x_0) + 2h^2f''(x_0) +
  \frac{4}{3}h^3f'''(x_0) + \frac{2h^4}{3}f''''(x_0) +
  \frac{4h^5}{15}f^{(5)}(\theta)\)
\item \(f(x_0-2h) = f(x_0) - 2hf'(x_0) + 2h^2f''(x_0) -
  \frac{4}{3}h^3f'''(x_0) + \frac{2h^4}{3}f''''(x_0) -
  \frac{4h^5}{15}f^{(5)}(\theta)\)
\end{itemize}
Then we take a linear combination of the terms with coefficients
\(a_1, a_2, a_3, a_4\) to find
\begin{align*}
  &a_1f(x_0+h) + a_2f(x_0-h) + a_3(fx_0+2h) + a_4f(x_0-2h)\\
  &=(a_1+a_2+a_3+a_4)f(x_0)\\
  &+ (a_1-a_2+2a_3-2a_4)hf'(x_0) + \\
  &+(\frac{1}{2}a_1 + \frac{1}{2}a_2 + 2a_3 + 2a_4)h^2f''(x_0)\\
  &+(\frac{1}{6}a_1 - \frac{1}{6}a_2 + \frac{4}{3}a_3 - \frac{4}{3}a_4)h^3f'''(x_0)\\
  &+(\frac{1}{24}a_1 + \frac{1}{24}a_2 + \frac{2}{3}a_3 + \frac{2}{3}a_4)h^4f''''(x_0)\\
  &+(\frac{1}{120}a_1 - \frac{1}{120}a_2 + \frac{4}{15}a_3 - \frac{4}{15}a_4)f^{(5)}(\theta)
\end{align*}
This leaves us with the system of equations
\begin{align*}
  (a_1+a_2+a_3+a_4) &= 0\\
  (a_1-a_2+2a_3-2a_4)h &= 1\\
  (\frac{1}{2}a_1 + \frac{1}{2}a_2 + 2a_3 + 2a_4)h^2 &= 0\\
  (\frac{1}{6}a_1 - \frac{1}{6}a_2 + \frac{4}{3}a_3 - \frac{4}{3}a_4)h^3 &= 0\\
\end{align*}
Solving this via Matlab, we find \(a_1=\frac{2}{3h},
a_2=\frac{-2}{3h}, a_3=\frac{-1}{12h}, a_4=\frac{1}{12h}\). This means
that
\[f'(x_0) = \frac{\frac{2}{3}f(x_0+h) - \frac{2}{3}f(x_0-h)
  - \frac{1}{12}f(x_0+2h) + \frac{1}{12}f(x_0-2h)}{h}
+ \frac{-h^4}{30}f^{(5)}(\theta)\]
And the convergence is on the order of \(h^4\).
  

% Question 4
\section{}
We want to find the error term of
\[f''(x_0) \approx \frac{-f(x_0+3h)+4f(x_0+2h)-5f(x_0+h)+2f(x_0)}{h^2}\]
First we find the Taylor Series approximation of 
\begin{itemize}
\item \(f(x_0+3h)\)
\item \(f(x_0+2h)\)
\item \(f(x_0+h)\)
\end{itemize}
For this problem we need to find four terms, with the fourth term
being the error term.  Then we plug these into our formula above and
find that
\begin{align*}
    -f(x_0+3h)+4f(x_0+2h)-5f(x_0+h)+2f(x_0)  &= (2-1+4-5)f(x_0) \\
    &+ (-3+8-5)hf(x_0) \\
    &+ (\frac{-9}{2}+8+\frac{-5}{2})h^2f''(x_0)\\
    &+ (\frac{16}{3}+\frac{27}{6}+\frac{-5}{6})h^3f^{(3)}(x_0)\\
    &+ (\frac{-81}{24}+\frac{8}{3}+\frac{-5}{24})h^4f^{(4)}(\theta)
\end{align*}
And thus
\[\frac{-f(x_0+3h)+4f(x_0+2h)-5f(x_0+h)+2f(x_0)}{h^2} = f''(x_0) - \frac{11}{12}f^{(4)}(\theta)\]


% Question 5
\section{}
We start with the original matrix and a pivot vector P to keep track
of pivoting.
\[
A=
\begin{bmatrix}
  1 & 1 &-1 & 2\\
  2 & 2 & 4 & 5\\
  1 &-1 & 1 & 7\\
  2 & 3 & 4 & 6
\end{bmatrix}
, \quad
P=
\begin{bmatrix}
  1\\
  2\\
  3\\
  4\\
\end{bmatrix}
\]
First we look to pivot to put the maximum absolute value of the first
column in the \(a_{1,1}\) position. This leaves us with
\[
\begin{bmatrix}
  2 & 2 & 4 & 5\\
  1 & 1 &-1 & 2\\
  1 &-1 & 1 & 7\\
  2 & 3 & 4 & 6
\end{bmatrix}
, \quad
P=
\begin{bmatrix}
  2\\
  1\\
  3\\
  4\\
\end{bmatrix}
\]
Then we subtract half of the first row, from the second row, half of
the first row from the third row, and one of the first row from the
fourth row. This leaves us with zeros in the
\(a_{2,1},a_{3,1}\&a_{4,1}\) positions. We need to keep track of our
row operations, so we will utilize these zeros to track them for our L
matrix. This means the diagonal and above corresponds to the U matrix,
and below the diagonal (plus an identity matrix) corresponds to the L
matrix.
This leaves us with
\[
\begin{bmatrix}
  2  &  2  &  4  &  5\\
  1/2&  0  & -3  &-1/2\\
  1/2& -2  & -1  & 9/2\\
  1  &  1  &  0  &  1\\
\end{bmatrix}
\]
The we switch the second row and the third row, and add half of the
second row to the fourth row. This leaves us with
\[
\begin{bmatrix}
  2  &  2  &  4  &  5\\
  1/2& -2  & -1  & 9/2\\
  1/2&  0  & -3  &-1/2\\
  1  & -1/2& -1/2&13/4\\
\end{bmatrix}
, \quad
P=
\begin{bmatrix}
  2\\
  3\\
  1\\
  4\\
\end{bmatrix}
\]
The last step is to subtract one-sixth of the third row from the forth
row. We also convert the P vector to a P matrix which is the
equivalent of the sum of our row operations. The final answer is
\[
L=
\begin{bmatrix}
  1  &  0  &  0  &  0\\
  1/2&  1  &  0  &  0\\
  1/2&  0  &  1  &  0\\
  1  &-1/2 & 1/6 &  1
\end{bmatrix}
, \quad
U=
\begin{bmatrix}
  2  &  2  &  4  &  5\\
  0  & -2  & -1  & 9/2\\
  0  &  0  & -3  &-1/2\\
  0  &  0  &  0  &10/3\\
\end{bmatrix}
, \quad
P =
\begin{bmatrix}
  0 & 1 & 0 & 0\\
  0 & 0 & 1 & 0\\
  1 & 0 & 0 & 0\\
  0 & 0 & 0 & 1\\
\end{bmatrix}
\]
where \(PA=LU\).\\
We can now use this to solve a system \(Ax=b\) quite easily. We have
\[
b=
\begin{bmatrix}
  6\\
  21\\
  25\\
  23
\end{bmatrix}
\]
thus, we can simply perform \(y=L*(P*b)\) and \(x=Uy\) which is very
easy to do on triangular matrices thanks to backward and forward
substitution.  Saving some space, we have
\[
y=
\begin{bmatrix}
  21\\
  29/2\\
  -9/2\\
  10
\end{bmatrix}
, \quad
x=
\begin{bmatrix}
  2\\
  -1\\
  1\\
  3
\end{bmatrix}
\]


% Question 6
\section{}
\subsection*{a.}
First let's find the LDL' factorization.
\[
L=
\begin{bmatrix}
  1  &  0  &  0  & 0\\
  1/4&  1  &  0  & 0\\
  1/4&-1/11&  1  & 0\\
  1/4&-5/11&7/19 & 1
\end{bmatrix}
, \quad
D=
\begin{bmatrix}
  4  &  0  &  0  &  0\\
  0  &11/4 &  0  &  0\\
  0  &  0  &19/11&  0\\
  0  &  0  &  0  &56/19
\end{bmatrix}
\]
\subsection*{b.}
Now let's find the Choleskly factorization
\[
L=
\begin{bmatrix}
  2  & 0         & 0       & 0\\
  1/2& 1985/1197 & 0       & 0\\
  1/2& -409/2713 & 1309/996& 0\\
  1/2& -1197/1588& 475/981 & 2546/1483
\end{bmatrix}
\]

\end{document}
